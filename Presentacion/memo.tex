\documentclass[aspectratio=169]{beamer}

\usetheme[progressbar=frametitle]{metropolis}
\usepackage{appendixnumberbeamer}
\definecolor{MyBackground}{RGB}{255,255,255}
\setbeamercolor{background canvas}{bg=MyBackground}

\usepackage{booktabs}
\usepackage[scale=2]{ccicons}

\usepackage{pgfplots}
\usepgfplotslibrary{dateplot}

\usepackage[english]{babel}
\usepackage[utf8x]{inputenc}

\usepackage{xspace}
\newcommand{\themename}{\textbf{\textsc{metropolis}}\xspace}

\title{Diseño y análisis de redes homeostáticas adaptativas}
\subtitle{Grado de Ingeniería Informática}
% \date{\today}
\date{}
\author[Alberto Martínez Menéndez]{\textbf {Alberto Martínez Menéndez\\ \footnotesize Director: Manuel Gonzalez Bedía}}
\institute{Escuela de Ingeniería y Arquitectura}
\titlegraphic{\hfill\includegraphics[height=1.5cm]{Imagenes/UnizarLogo}}
\begin{document}

\maketitle

\begin{frame}{Tabla de contenidos}
  \setbeamertemplate{section in toc}[sections numbered]
  \tableofcontents[hideallsubsections]
\end{frame}

\section{Motivación}
\begin{frame}{Motivación}
\begin{itemize}
  \item {Agentes artificiales basados en \textbf{hábitos} en lugar de en \textbf{instrucciones}.}
  \item {Sistemas o modelos automantenidos (\textbf{Sistema homeostático}).}
\end{itemize}
  \vspace{1cm}
  \begin{columns}
    \begin{column}{0.47\textwidth}
      \includegraphics[width=0.9\textwidth,height=.5\textheight]{Imagenes/MirarCalle}
    \end{column}
    \begin{column}{0.5\textwidth}
      \includegraphics[width=0.9\textwidth,height=.5\textheight]{Imagenes/Camino}
    \end{column}
  \end{columns}
\end{frame}

\begin{frame}{Motivación}
  \begin{itemize}
    \item {Controladores con \textbf{plasticidad} neuronal.}
    \item {Comportamientos sociales entendidos como dos corrientes distintas:}
    \begin{itemize}
      \item {Entendidos desde una aproximación \textbf{aditiva}.}
      \item {Entendidos desde una aproximación \textbf{estructurada}.}
    \end{itemize}
  \end{itemize}
  \vspace{1cm}
  \begin{columns}
    \begin{column}{0.47\textwidth}
      \includegraphics[width=0.9\textwidth,height=.5\textheight]{Imagenes/Plasticidad}
    \end{column}
    \begin{column}{0.5\textwidth}
       \includegraphics[width=0.9\textwidth,height=.5\textheight]{Imagenes/Robots}
    \end{column}
  \end{columns}
\end{frame}

\section{Objetivos}
\begin{frame}{Objetivos}
  \begin{itemize}
  \item {Diseño de agentes artificiales homeostáticos capaces de desarrollar un comportamiento de \textbf{fototaxis}.}
  \item {Comprobar la robustez de los agentes artificiales en entornos colectivos según las capacidades sociales se hayan desarrollado de manera \textbf{aditivas} o \textbf{estructurales}.}
  \end{itemize}
\end{frame}

\begin{frame}{Objetivos: pasos seguidos}
  \begin{itemize}
  \item {Comprensión del concepto de homeostasis.}
  \item {Diseño e implementación de un agente artificial con comportamientos de fototaxis individual y plasticidad.}
  \item {Diseño e implementación de dos agentes artificiales (uno según cada corriente) con comportamientos de fototaxis colectiva y plasticidad.}
  \item {Ejecución de pruebas colectivas, análisis de los resultados y obtención de conclusiones.}
  \end{itemize}
\end{frame}

\section{Modelo}
\begin{frame}{Modelo}
\begin{block}{Agente artificial}
  \begin{columns}
    \begin{column}{0.47\textwidth}
        \begin{itemize}
            \item Dos sensores.
            \item Dos motores.
            \item Controlador basado en una CTRNN.
        \end{itemize}
    \end{column}
    \begin{column}{0.5\textwidth}
        \includegraphics[width=0.8\textwidth,height=.35\textheight]{Imagenes/AgentSchema}
    \end{column}
\end{columns}
\end{block}
\end{frame}

\begin{frame}{Modelo}
  \begin{block}{Controlador basado en una Red Neuronal Recurrente de Tiempo Continuo (CTRNN)}
    \begin{equation*}
    	\dot{y_{i}}= \frac{1}{\tau_{i}} * \left ( -y_{i}+\sum_{j=1}^{N}w_{ji}*\sigma \left ( g(y_{j} + \theta _{j}) \right ) + I_{i} \right ) \qquad i =1,2,...,N
    \end{equation*}
    \begin{columns}
      \begin{column}{0.47\textwidth}
          \begin{itemize}
            \item $\dot{y_{i}}$: nueva activación (estado) de la neurona i.
            \item $\tau_{i}$: constante de tiempo de la neurona i.
            \item $y_{i}$: activación actual de la neurona i.
            \item $w_{ji}$: peso de la conexión (sináptico) entre las neuronas i y j.
          \end{itemize}
      \end{column}
      \begin{column}{0.5\textwidth}
        \begin{itemize}
          \item $\sigma ()$: función sigmoide de activación.
          \item $y_{j}$: activación actual de la neurona j.
          \item $\theta_{j}$: bias de la neurona j.
          \item g: ganancia de la neurona i.
          \item $I_{i}$: entrada de la neurona i.
        \end{itemize}
      \end{column}
    \end{columns}
  \end{block}
\end{frame}

\begin{frame}{Modelo}
  \begin{columns}
    \begin{column}{0.47\textwidth}
      \vspace{1cm}
      \begin{equation*}
        \sigma (x)=\frac{1}{1+e^{-x}}
      \end{equation*}
        \begin{itemize}
          \item Recordamos: $x = g(y_{j} + \theta_{j})$.
          \item Devuelve valores entre 0 y 1.
          \item $\theta$ (bias): desplazamiento de la función.
          \item g (ganancia): inclinación de la función.
        \end{itemize}
    \end{column}
    \begin{column}{0.5\textwidth}
      \includegraphics[width=1.0\textwidth,height=.50\textheight]{Imagenes/Sigmoid}
    \end{column}
  \end{columns}
  \begin{itemize}
    \item Se han seleccionado estas redes porque:
    \begin{itemize}
      \item Permiten simular cualquier sistema dinámico.
      \item Son las más fieles al funcionamiento biológico de las activaciones neuronales.
    \end{itemize}
  \end{itemize}
\end{frame}

\begin{frame}{Modelo}
\begin{block}{Homeostasis}
\begin{itemize}
  \item "A Design for a Brain"(1952), William Ashby
  \item Matematiza la homeostasis, la cual viene de la fisiología.
  \item Seres vivos tienen variables internas que deben mantenerse dentro de unos límites.
  \item Estado de supervivencia (Ultraestabilidad): ninguna variable se encuentra fuera de los límites de estabilidad.
  \item \textbf{Regulación homeostática}: mecanismos que permiten esta estabilidad.
  \item Permite la generación de comportamiento adaptativo autoinducido mediante el mantenimiento de la estabilidad interna.
\end{itemize}
\end{block}
\end{frame}

\begin{frame}{Modelo}
  \begin{columns}
    \begin{column}{0.5\textwidth}
        \begin{itemize}
          \item Se busca la estabilidad en la \textbf{conducta} de los agentes.
          \item Mecanismos homeostáticos en los agentes: \textbf{plasticidad}.
          \item Modificación de los pesos sinápticos para fortalecer o debilitar una conexión interneuronal.
          \item Basados en la "Teoría Hebbiana" (1949), Donal Hebb.
          \item La plasticidad de una de nuestras neuronas artificiales puede seguir cuatro reglas:
        \end{itemize}
    \end{column}
    \begin{column}{0.5\textwidth}
      \includegraphics[width=0.8\textwidth,height=.50\textheight]{Imagenes/Neurona}
    \end{column}
  \end{columns}

\end{frame}

\begin{frame}{Modelo}

R0: Sin plasticidad:
\begin{equation*}
 \centering
 \Delta w_{ij}= 0
\end{equation*}

R1: Aprendizaje Hebbiano acotado:
\begin{equation*}
 \centering
 \Delta w_{ij}= \delta n_{ij} p_{j} z_{i} z_{j}
\end{equation*}

R2: Potenciación o depresión amortiguadas de la neurona presináptica cuando la eficacia sináptica es muy alta o muy baja:
\begin{equation*}
 \centering
 \Delta w_{ij}= \delta n_{ij} p_{j} (z_{i} - z_{ij}^{o}) z_{j}
\end{equation*}

R3: Potenciación o depresión amortiguadas de la neurona postsináptica cuando la eficacia sináptica es muy alta o muy baja:
\begin{equation*}
 \centering
 \Delta w_{ij}= \delta n_{ij} p_{j} z_{i} (z_{j} - z_{ij}^{o})
\end{equation*}
\end{frame}

\begin{frame}{Modelo}
\begin{block}{Computación evolutiva}
  \begin{itemize}
    \item Utilizado para entrenar a los agentes (\textbf{Algoritmo genético}).
    \item Agentes codificados como vector de reales y enteros.
  \end{itemize}
  \begin{figure}
    \centering
  \includegraphics[width=0.7\textwidth,height=.3\textheight]{Imagenes/vector0}
\end{figure}
\end{block}
\end{frame}

\begin{frame}{Modelo}
\onslide{Funcionamiento:}
  \begin{enumerate}
    \item Generación de \textbf{población inicial} (primera generación).
    \item Evaluación de la generación mediante la \textbf{función fitness}.
    \item \textbf{Selección} de los mejores individuos de la generación.
    \item Creación de una nueva generación a partir de los individuos seleccionados (mediante \textbf{recombinaciones} y \textbf{mutaciones}).
    \item Volver al punto 2 hasta que el mejor individuo de la generación actual cumpla con las expectativas.
  \end{enumerate}

  \onslide{Algunas anotaciones sobre el algoritmo genético:}
  \begin{itemize}
    \item Población inicial de \textbf{60} candidatos.
    \item Función de selección de torneo.
    \item Función de recombinación uniforme.
    \item Función de mutación básica.
  \end{itemize}
\end{frame}

\section{Agente individual}
\subsection{Controlador}
\begin{frame}{Agente individual: Controlador}
  \begin{itemize}
    \item Agente con capacidades de fototaxis individual y plasticidad.
    \item Dos sensores luminosos y dos motores.
    \item Simetría en las ganancias de los motores y de los sensores.
    \item Prueba de concepto. Comprobar el correcto funcionamiento de la implementación de la fototaxis y la plasticidad.
  \end{itemize}
  \begin{figure}
    \centering
  \includegraphics[width=0.25\textwidth,height=0.5\textheight]{Imagenes/Agent0Controller}
\end{figure}
\end{frame}

\subsection{Función fitness}
\begin{frame}{Agente individual: Función fitness}
  \begin{itemize}
    \item Un agente individual situado en (0,0).
    \item De forma progresiva aparecen 6 luces a una distancia determinada del agente.
    \item Cada luz está encendida un número de ciclos aleatorio [1200, 2000].
    \item Solamente una luz encendida al mismo tiempo.
    \item La fitness está compuesta por tres factores:
    \begin{enumerate}
      \item \textbf{Fd}: Que para cada luz, el agente acabe más cerca de ella que cuando se encendió.
      \item \textbf{Fp}: Que para cada luz, el agente permanezca alrededor de ella el mayor número de ciclos posibles.
      \item \textbf{Fh}: Que para cada luz, el máximo número de neuronas del agente se mantengan estables.
    \end{enumerate}
  \end{itemize}
\end{frame}

\begin{frame}{Agente individual: Función fitness}
\begin{itemize}

  \item Máxima fitness alcanzada: \textbf{0.81}.
\end{itemize}

\begin{equation*}
 \centering
  fitness = (0.34Fd + 0.54Fp + 0.12Fh)
\end{equation*}
\noindent\rule{14cm}{0.4pt}

\begin{equation*}
  Fd=\begin{cases}
0.0 & \text{ IF }\quad D_{f} > D_{i}  \\
1 - (D_{f} / D_{i}) & \text{ IF }\quad D_{f} \leq D_{i} \\
\end{cases}
\end{equation*}
\noindent\rule{14cm}{0.4pt}

\begin{equation*}
\centering
Fp = \frac{\text{Nº ciclos cerca de la luz}}{\text{Nº de ciclos que la luz ha estado encendida}}
\end{equation*}
\noindent\rule{14cm}{0.4pt}

\begin{equation*}
\centering
Fh = \frac{\text{Nº neuronas que no han perdido estabilidad}}{\text{Nº de neuronas del agente}}
\end{equation*}
\end{frame}

\subsection{Experimentos}
\begin{frame}{Agente individual: Experimentos}
  \begin{figure}
    \centering
  \includegraphics[width=0.7\textwidth,height=0.7\textheight]{Imagenes/Agente0Trayectoria}
\end{figure}
\end{frame}

\section{Agentes colectivos}
\begin{frame}{Agentes colectivos}
  \begin{itemize}
    \item Agentes artificiales con capacidades de fototaxis colectiva y plasticidad.
    \item Restricción colectiva: no más de \textbf{3} agentes pueden estar situados cerca de una luz al mismo tiempo.
    \item Dos sensores luminosos, dos sensores de agentes y dos motores.
    \item Simetría en las ganancias de los motores y de los sensores.
    \item Agente de tipo 1 (\textbf{aditivo}) $|$ Agente de tipo 2 (\textbf{estructural}).
    \item Un grupo de 5 agentes de tipo 1 y un grupo de 5 agentes de tipo 2.
  \end{itemize}
\end{frame}

\subsection{Controladores}
\begin{frame}{Agentes colectivos: Controladores}
  \begin{columns}
    \begin{column}{0.5\textwidth}
      \begin{block}{Agente 1 (aditivo)}
        \vspace{0.5cm}
      \includegraphics[width=0.8\textwidth,height=.50\textheight]{Imagenes/Agent1Controller}
    \end{block}
    \end{column}
    \begin{column}{0.5\textwidth}
      \begin{block}{Agente 2 (estructural)}
        \vspace{0.5cm}
      \includegraphics[width=0.8\textwidth,height=.50\textheight]{Imagenes/Agent2Controller}
      \end{block}
    \end{column}
  \end{columns}
\end{frame}

\subsection{Función fitness}
\begin{frame}{Agentes colectivos: Función fitness}
  \begin{itemize}
    \item Un grupo de 5 agentes de tipo 1 o tipo 2 situado en torno al (0,0).
    \item De forma progresiva aparecen 6 luces a una distancia mínima del centroide de los agentes.
    \item Cada luz está encendida un número de ciclos aleatorio [1200, 2000].
    \item Solamente una luz encendida al mismo tiempo.
    \item La fitness está compuesta por tres factores:
    \begin{enumerate}
      \item \textbf{Fd}: Que para cada luz, los agentes acaben más cerca de ella que cuando se encendió.
      \item \textbf{Fp colectiva}: Que para cada luz, se respete correctamente la restricción colectiva añadida.
      \item \textbf{Fh}: Que para cada luz, cada agente mantenga estables el mayor número posible de neuronas.
    \end{enumerate}
  \end{itemize}

\end{frame}

\begin{frame}{Agentes colectivos: Función fitness}

  \begin{columns}
      \begin{column}{0.5\textwidth}
        \begin{block}{Agente 1 (aditivo)}
          \begin{itemize}
            \item Máxima fitness alcanzada: \textbf{0.472}.
          \end{itemize}
      \end{block}
      \end{column}
      \begin{column}{0.5\textwidth}
        \begin{block}{Agente 2 (estructural)}
          \begin{itemize}
            \item Máxima fitness alcanzada: \textbf{0.591}.
          \end{itemize}
        \end{block}
      \end{column}
  \end{columns}
  \vspace{-0.2cm}

  \begin{equation*}
   \centering
  	fitness = (0.44Fd + 0.44Fp + 0.12Fh)
  \end{equation*}
  \noindent\rule{14cm}{0.4pt}

  \begin{equation*}
    Fd = \frac{\sum_{k=1}^{5}(\frac{\sum_{j=1}^{NºLuces}Fd_{kj}}{NºLuces})}{5} \qquad \Rightarrow \qquad Fd_{kj}=\begin{cases}
    0.0 & \text{ IF }\quad D_{fkj} > D_{ikj}  \\
    1 - (D_{fkj} / D_{ikj}) & \text{ IF }\quad D_{fkj} \leq D_{ikj} \\
    \end{cases}
  \end{equation*}
  \noindent\rule{14cm}{0.4pt}

  \begin{equation*}
   \centering
  		Fp = \frac{\sum_{i= 1}^{\text{NºLuces}}\text{Contador}_{i}}{(\sum_{i =1}^{\text{NºLuces}}\text{Ciclos encendida}_{i}) * 3}
  \end{equation*}
  \noindent\rule{14cm}{0.4pt}

  \begin{equation*}
   \centering
  		Fh = \frac{\sum_{i = 1}^{Nº Luces}\frac{\sum_{k=1}^{5}(\frac{\text{Nº neuronas que no han perdido estabilidad}_{ki}}{\text{Nº de neuronas del agente}_{ki}})}{5}}{Nº Luces}
  \end{equation*}
\end{frame}

\subsection{Experimentos}
\begin{frame}{Agentes colectivos: Experimentos}
\begin{itemize}
  \item Función similar función fitness colectiva.
  \item Se ha buscado comparar a los dos grupos de agentes respecto a:
  \begin{itemize}
    \item Sus trayectorias y el cumplimiento de la fototaxis colectiva.
    \item Posibles patrones de activación en sus neuronas.
    \item La robustez de los agentes ante ruido en los sensores.
    \item La robustez de los agentes ante ruido en la plasticidad.
  \end{itemize}
\end{itemize}
\end{frame}

\begin{frame}{Agentes colectivos: Experimentos (Trayectorias)}
  \begin{columns}
    \begin{column}{0.50\textwidth}
      \begin{block}{Agente 1 (aditivo)}
        \vspace{0.5cm}
      \includegraphics[width=1.1\textwidth,height=.6\textheight]{Imagenes/Agente1Trayectoria}
    \end{block}
    \end{column}
    \begin{column}{0.5\textwidth}
      \begin{block}{Agente 2 (estructural)}
        \vspace{0.5cm}
      \includegraphics[width=1.1\textwidth,height=.6\textheight]{Imagenes/Agente2Trayectoria}
      \end{block}
    \end{column}
  \end{columns}
\end{frame}

\begin{frame}{Agentes colectivos: Experimentos (Activaciones neuronales)}
  \begin{columns}
    \begin{column}{0.47\textwidth}
      \begin{block}{Agente 1 (aditivo)}
        \vspace{0.5cm}
      \includegraphics[width=1.0\textwidth,height=.6\textheight]{Imagenes/Neurona0}
    \end{block}
    \end{column}
    \begin{column}{0.5\textwidth}
      \begin{block}{Agente 2 (estructural)}
        \vspace{0.5cm}
      \includegraphics[width=1.0\textwidth,height=.6\textheight]{Imagenes/Neurona4}
      \end{block}
    \end{column}
  \end{columns}
\end{frame}

\begin{frame}{Agentes colectivos: Experimentos (Robustez en los sensores)}
  \begin{itemize}
    \item Comprobar si la fitness se ve afectada por el ruido en los sensores.
    \item Para cada grupo, \textbf{30} mediciones de la fitness para cada nivel de ruido.
    \item Factores de ruido probados: 0.0, 0.2, 0.34 y 0.44.
  \end{itemize}
  % \vspace{-0.5cm}
  \begin{columns}
    \begin{column}{0.47\textwidth}
      \begin{block}{Agente 1 (aditivo)}
      \includegraphics[width=1.0\textwidth,height=.6\textheight]{Imagenes/BoxPlot1}
    \end{block}
    \end{column}
    \begin{column}{0.5\textwidth}
      \begin{block}{Agente 2 (estructural)}
      \includegraphics[width=1.0\textwidth,height=.6\textheight]{Imagenes/BoxPlot2}
      \end{block}
    \end{column}
  \end{columns}
\end{frame}

\begin{frame}{Agentes colectivos: Experimentos (Robustez en los sensores)}
  \begin{itemize}
    \item Test \textbf{t de Student}.
    \item Hipótesis: ¿Pueden considerarse las medias de las distribuciones iguales?
  \end{itemize}
  \begin{columns}
    \begin{column}{0.47\textwidth}
      \begin{block}{Agente 1 (aditivo)}
        \begin{table}
         \begin{tabular}{lr}
           \toprule
           Comparación & p-valor\\
           \midrule
           Sin ruido - 0.2 & 0.075\\
           Sin ruido - 0.34 & 0.39\\
           Sin ruido - 0.44 & 0.54\\
           \bottomrule
         \end{tabular}
       \end{table}
    \end{block}
    \end{column}
    \begin{column}{0.5\textwidth}
      \begin{block}{Agente 2 (estructural)}
        \begin{table}
         \begin{tabular}{lr}
           \toprule
           Comparación & p-valor\\
           \midrule
           Sin ruido - 0.2 & 0.17\\
           Sin ruido - 0.34 & 0.56\\
           Sin ruido - 0.44 & 0.31\\
           \bottomrule
         \end{tabular}
       \end{table}
      \end{block}
    \end{column}
  \end{columns}
\end{frame}


\begin{frame}{Agentes colectivos: Experimentos (Robustez en la plasticidad)}
  \begin{itemize}
    \item Comprobar si la fitness se ve afectada por el ruido en los parámetros de la plasticidad.
    \item Para cada grupo, \textbf{30} mediciones de la fitness para cada nivel de ruido.
    \item Dos pruebas: ruido en el ritmo de plasticidad y ruido en los pesos sinápticos.
    \item Factores de ruido probados: 0.0, 0.2, 0.34 y 0.44.
  \end{itemize}
\end{frame}

\begin{frame}{Agentes colectivos: Experimentos (Robustez en la plasticidad)}
  \begin{itemize}
    \item Ruido en el parámetro \textbf{ritmo de plasticidad}.
    \begin{columns}
      \begin{column}{0.47\textwidth}
        \begin{block}{Agente 1 (aditivo)}
        \includegraphics[width=1.0\textwidth,height=.6\textheight]{Imagenes/BoxPlotNP1}
      \end{block}
      \end{column}
      \begin{column}{0.5\textwidth}
        \begin{block}{Agente 2 (estructural)}
        \includegraphics[width=1.0\textwidth,height=.6\textheight]{Imagenes/BoxPlotNP2}
        \end{block}
      \end{column}
    \end{columns}
  \end{itemize}
\end{frame}

\begin{frame}{Agentes colectivos: Experimentos (Robustez en la plasticidad)}
  \begin{itemize}
    \item Test \textbf{t de Student}.
    \item Hipótesis: ¿Pueden considerarse las medias de las distribuciones iguales?
  \end{itemize}
  \begin{columns}
    \begin{column}{0.47\textwidth}
      \begin{block}{Agente 1 (aditivo)}
        \begin{table}
         \begin{tabular}{lr}
           \toprule
           Comparación & p-valor\\
           \midrule
           Sin ruido - 0.2 & 0.15\\
           Sin ruido - 0.34 & 0.52\\
           Sin ruido - 0.44 & 0.25\\
           \bottomrule
         \end{tabular}
       \end{table}
    \end{block}
    \end{column}
    \begin{column}{0.5\textwidth}
      \begin{block}{Agente 2 (estructural)}
        \begin{table}
         \begin{tabular}{lr}
           \toprule
           Comparación & p-valor\\
           \midrule
           Sin ruido - 0.2 & 0.001\\
           Sin ruido - 0.34 & 0.0347\\
           Sin ruido - 0.44 & 0.00041\\
           \bottomrule
         \end{tabular}
       \end{table}
      \end{block}
    \end{column}
  \end{columns}
\end{frame}

\begin{frame}{Agentes colectivos: Experimentos (Robustez en la plasticidad)}
  \begin{itemize}
    \item Ruido en el parámetro \textbf{peso sináptico}.
    \begin{columns}
      \begin{column}{0.47\textwidth}
        \begin{block}{Agente 1 (aditivo)}
        \includegraphics[width=1.0\textwidth,height=.6\textheight]{Imagenes/BoxPlotNW1}
      \end{block}
      \end{column}
      \begin{column}{0.5\textwidth}
        \begin{block}{Agente 2 (estructural)}
        \includegraphics[width=1.0\textwidth,height=.6\textheight]{Imagenes/BoxPlotNW2}
        \end{block}
      \end{column}
    \end{columns}
  \end{itemize}
\end{frame}

\begin{frame}{Agentes colectivos: Experimentos (Robustez en la plasticidad)}
  \begin{itemize}
    \item Test \textbf{t de Student}.
    \item Hipótesis: ¿Pueden considerarse las medias de las distribuciones iguales?
  \end{itemize}
  \begin{columns}
    \begin{column}{0.47\textwidth}
      \begin{block}{Agente 1 (aditivo)}
        \begin{table}
         \begin{tabular}{lr}
           \toprule
           Comparación & p-valor\\
           \midrule
           Sin ruido - 0.2 & 0.000000011\\
           Sin ruido - 0.34 & 0.00000002\\
           Sin ruido - 0.44 & 0.000001744\\
           \bottomrule
         \end{tabular}
       \end{table}
    \end{block}
    \end{column}
    \begin{column}{0.5\textwidth}
      \begin{block}{Agente 2 (estructural)}
        \begin{table}
         \begin{tabular}{lr}
           \toprule
           Comparación & p-valor\\
           \midrule
           Sin ruido - 0.2 & $\sim 0$\\
           Sin ruido - 0.34 & $\sim 0$\\
           Sin ruido - 0.44 & $\sim 0$\\
           \bottomrule
         \end{tabular}
       \end{table}
      \end{block}
    \end{column}
  \end{columns}
\end{frame}

\section{Conclusiones}
\begin{frame}{Conclusiones}
\begin{block}{Comportamientos o conducta externa}
  \begin{enumerate}
    \item Los mecanismos de doble red neuronal (Agente 1), hacen que tengan menor eficacia en el alcance de los objetivos (fitness sustancialmente menor).
    \item En los agentes de tipo 1, las trayectorias tienen un caracter más errático.
    \item Ambos tipos de agente no se ven afectados por las perturbaciones en los sensores.
  \end{enumerate}
\end{block}
\end{frame}

\begin{frame}{Conclusiones}
\begin{block}{Estructura o robustez interna}
  \begin{enumerate}
    \item Las perturbaciones en los pesos sinápticos afectan significativamente al rendimiento de los dos agentes. Las estructuras neuronales obtenidas por evolución son sensibles a cambios.
    \item Los agentes de tipo 1 pierden estabilidad con más frecuencia pero se recuperan rápido. Los agentes de tipo 2 son más estables, pero tardan más ciclos en volver a la estabilidad cuando salen.
    \item Las perturbaciones en los ritmos de plasticidad afectan al rendimiento de los agentes de tipo 2, pero no a los de tipo 1. Aunque a nivel de comportamiento, el Agente 2 presente características más robustas,
    mantiene un nivel crítico en su estructura homeostática.
  \end{enumerate}
\end{block}
\end{frame}

\begin{frame}{Conclusiones}
\begin{block}{Discusión}
  \vspace{-0.2cm}
  \begin{itemize}
    \item Debate de Psicología o Estudios del desarrollo cognitivo $=>$ abordarlo desde el \textbf{modelado cognitivo mínimo} (agentes artificiales).
    \item Busca modelos que, con mínimos recursos, puedan materializar las bases de capacidades cognitivas más elaboradas.
    \item Concentrarse en una versión simplificada de un fenómeno sirve como herramienta para pensar sobre dicho fenómeno.
    \item \textbf{Metodología sintética}: uso de robots, modelos de simulación, bots, etc. como plataformas experimentales para la investigación sobre cuestiones relacionadas con la inteligencia.
    \item Permite un acercamiento a cualquier ciencia desde la ingeniería. Útil para contrastar hipótesis.
    \item \textbf{Modelos mínimos}: su simpleza permite que sean estudiados de manera completa.
  \end{itemize}
\end{block}
\end{frame}

{\setbeamercolor{palette primary}{fg=black, bg=white}
\begin{frame}[standout]
  Muchas gracias por su atención.
\end{frame}
}

\begin{frame}[fragile]{Cronograma}
  \includegraphics[width=1.0\textwidth,height=.45\textheight]{Imagenes/Cronograma}
\end{frame}

\begin{frame}[fragile]{}
  \begin{columns}
    \begin{column}{0.47\textwidth}
      \includegraphics[width=1.0\textwidth,height=.5\textheight]{Imagenes/CustomHomeostat}
    \end{column}
    \begin{column}{0.5\textwidth}
      \includegraphics[width=1.0\textwidth,height=.5\textheight]{Imagenes/Decaimiento}
    \end{column}
  \end{columns}
\end{frame}
\end{document}
