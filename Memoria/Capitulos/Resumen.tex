\begin{center}
{\Large \bfseries Diseño y análisis de redes homeostáticas adaptativas}

\vspace{1cm}
{\Large \bfseries RESUMEN}

\vspace{2.5cm}
\end{center}

Conseguir dotar a un agente artificial con la capacidad de adaptabilidad presente en los seres vivos le proporcionaria las habilidades necesarias para que pudiera realizar tareas para las que no ha sido entrenado o
para realizar tareas para las que se le ha entrenado pero en un entorno cambiante o con incertidumbre. Los mecanismos necesarios para conseguir estas propiedades se llaman mecanismos homeostáticos.

En este trabajo se ha buscado diseñar e implementar un agente homeostático basado en redes neuronales recurrentes de tiempo continuo. Este agente se corresponde con el mejor candidato de entre una cierta población de candidatos,
el cual ha sido seleccionado mediante un algorítmo genético elitista que aplica una sencilla prueba a cada uno de ellos para conseguir un ranking que clasifica a la población según la puntuación obtenida en base a unos parámetros
que son medidos para cada candidato que realiza la prueba. El agente logra el comportamiento homeostático a traves de características de plasticidad, que le permiten modificar sus variables internas para alcanzar la estabilidad
buscada.

Una vez obtenido un agente con el comportamiento homeostático deseado se han ejecutado dos pruebas que han permitido obtener conclusiones sobre el comportamiento de este tipo de agentes:

....... TODO
