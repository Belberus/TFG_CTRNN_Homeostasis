\begin{center}
{\Large \bfseries Diseño y análisis de redes homeostáticas adaptativas}

\vspace{1cm}
{\Large \bfseries RESUMEN}

\vspace{2.5cm}
\end{center}

Conseguir dotar a un agente artificial con la capacidad de adaptabilidad presente en los seres vivos le proporcionaria las habilidades necesarias para que pudiera realizar tareas para las que no ha sido entrenado o
para realizar tareas para las que se le ha entrenado pero en un entorno cambiante o con incertidumbre. Los mecanismos necesarios para conseguir estas propiedades se llaman mecanismos homeostáticos.

En este trabajo se ha buscado diseñar e implementar un agente homeostático basado en redes neuronales recurrentes de tiempo continuo con capacidad de fototaxis (búsqueda y acercamiento a una serie de luces en un espacio).
Este agente se corresponde con el mejor candidate de entre una cierta población de candidatos, el cual ha sido seleccionado mediante un algorítmo genético. El agente desarrolla el comportamiento homeostático a través de
la evolución y de sus mecanismos de plasticidad, que le permiten modificar sus variables internas para alcanzar un estado estable y mantenerse en el mismo.

Una vez obtenido el agente con las propiedades de fototaxis y plasticidad, se ha buscado dotar al mismo de comportamientos sociales para estudiar como interacciona en situaciones o pruebas donde hay más de un agente
involucrado. El comportamiento social se ha añadido al agente siguiendo dos aproximaciones diferentes. La primera, desarrollada por () defiende que el comportamiento social es una característica que se tiene de manera
innata pero que se va entrenando con el tiempo. La segunda, desarrollada por () defiende que el comportamiento social es una característica añadida a la red neuronal e independiente de la misma. El objetivo es comparar
estas dos aproximaciones y sacar conclusiones sobre ello creando un experimento en el que varios agentes tienen que interactuar de manera social para la obtención de un objetivo.
