\begin{center}
{\Large \bfseries Diseño y análisis de redes homeostáticas adaptativas}

\vspace{1cm}
{\Large \bfseries RESUMEN}

\vspace{1.5cm}
\end{center}

Conseguir dotar a un agente artificial con la capacidad de adaptabilidad presente en los seres vivos le proporcionaría habilidades necesarias para realizar tareas para las que no ha sido entrenado o
para las que se le ha entrenado pero en un entorno cambiante o con incertidumbre. En este trabajo nos interesaremos, desde una perspectiva ingenieril, por unos de los sistemas de autoregulación presentes
en los organismos vivos conocidos como mecanismos homeostáticos, que están en la base de ciertas capacidades de adaptación.

En particular, se ha partido del diseño e implementación de un agente homeostático basado en redes neuronales recurrentes de tiempo continuo con capacidad de fototaxis (búsqueda y acercamiento a una serie de luces en un espacio).
Este agente se corresponde con el mejor candidato de una cierta población, seleccionado mediante un algorítmo genético. La peculiaridad del mismo, es que despliega comportamiento homeostático a través de mecanismos de plasticidad,
que le permiten modificar sus variables internas para alcanzar un estado estable y mantenerse en el mismo aunque se vea sometido a perturbaciones externas.

Una vez obtenido el agente con las propiedades de fototaxis y plasticidad homeostática, se ha buscado dotar al mismo de capacidades para que emerjan comportamientos sociales con el fin de analizar cómo interacciona en situaciones donde hay más de un agente
involucrado. El comportamiento social se ha añadido siguiendo dos aproximaciones diferentes. La primera, asume que las capacidades del agente son aditivas. Es decir, que el agente tras ser capaz de navegar y orientarse
hacia un foco de luz (o comida), necesitaría una nueva estructura neuronal con la que codificar otra capacidad novedosa (en este caso, coordinación social). Se modelará el sistema con dos mecanismos independientes en el controlador
del sistema, uno para la navegación y otro para el aspecto social. La segunda, defiende que el comportamiento social es una característica estructural del sistema. Por tanto, una nueva capacidad atraviesa (en ocasiones se
dice que ``percola'') las capacidades previas, reestructurando el controlador neuronal del agente en su conjunto. El objetivo es comparar estas dos aproximaciones y sacar conclusiones sobre ello mediante experimentos en los que
varios agentes tienen que interactuar de manera social para la consecución de un objetivo.
