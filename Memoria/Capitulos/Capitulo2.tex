\chapter{Estado del arte}
\pagenumbering{arabic}
\subsubsection{Hábitos}
En nuestro comportamiento diario desplegamos numerosos patrones de conducta que se han ido fortaleciendo con la experiencia, de un modo continuado, y sostenidos por la repetición. Por ejemplo, mirar hacia la izquierda o hacia la derecha antes de cruzar la calle, atarse los cordones de los zapatos o simplemente caminar, pueden entenderse como patrones anidados de coordinación sensomotora, consolidados en nuestra conducta por una historia de refuerzos: A estos esquemas de conducta les llamamos hábitos.

Durante un tiempo, los hábitos fueron la piedra angular en psicología hasta el surgimiento del computador en la década de los sesenta del siglo pasado. Desde entonces, los modelos psicológicos que se extendieron y que funcionaron como inspiración entre los ingenieros que desarrollaban máquinas inteligentes, asumieron típicamente modelos internos de procesamiento simbólico-lingüístico, relegando el concepto de hábito al de un simple automatismo de estímulo-respuesta, lo que supone una simplificación muy alejada de la esencia y la génesis de los hábitos.

¿A qué o cómo definiríamos un hábito en sentido estricto? Podríamos decir que es un patrón automantenido de conducta. Decimos auto-mantenido porque la estabilidad de la conducta está acoplada con la estabilidad de los mecanismos que la generan. El hábito ``llama'' a realizar una conducta y la conducta refuerza el hábito.

En este trabajo, estamos interesados en analizar hábitos desde un punto de vista operativo para implementar agentes virtuales cuyo comportamiento se base precisamente en ellos. De modo que nuestras implementaciones deberán cumplir aspectos como que: (i) los hábitos no presupongan una prioridad causal entre percepción y acción, sino que integren a las dos; (ii) los hábitos sean plásticos y maleables (a diferencia de la rigidez que acompañan a las nociones de arco reflejo); (iii) los hábitos provean un sentido concreto de automantenimiento en el nivel de la conducta (es decir, sean formas de acción auto-reforzadas).

\subsubsection{Agentes artificiales basados en hábitos: Operativización de la noción de homeostasis}
Para modelar este tipo de hábitos en agentes artificiales que se desenvuelvan en entornos no conocidos y desarrollen tareas (es decir, para programar este tipo de patrones que se automantienen y fortalecen por la experiencia de practicarlos) nos interesaremos por las nociones que, en la década de los cuarenta, William Ross Ashby (1903-1972), desarrolló para operativizar e implementar en forma de artefacto, el concepto de homeostasis.

En general, la homeostasis es la capacidad de un organismo para mantener constante las propiedades físico-químicas de su medio interno. Es una habilidad que permite que su situación físico-química característica se conserve constante dentro de unos límites, aunque en el exterior existan fuentes que puedan ser motivo de alteración. Por ejemplo, el cuerpo humano, frente a fuentes de cambio externas, moviliza mecanismos de autorregulación (como el sistema nervioso central, sistema endocrino, sistema respiratorio, sistema circulatorio, etc.) para que se mantengan de forma constante sus condiciones de vida.

Aunque tradicionalmente es un término vinculado a la biología o la fisiología, otras ciencias y técnicas alejadas se han interesado por el fenómeno homeostático y han adoptado también este concepto a sus intereses. En su adopción técnica, la homeostasis es simplemente el rasgo de sistemas autorregulados que consiste en la capacidad para mantener ciertas variables en un estado estacionario, de equilibrio dinámico o dentro de ciertos límites, cambiando parámetros de su estructura interna. En este sentido, Ashby fue el primero que capturó operativamente estas ideas y construyó un homeostato artificial para demostrar las características de comportamiento de, lo que él llamó, un sistema ultraestable. El homeostato de Ashby respondía de manera aleatoria a las desviaciones de ciertos valores en sus parámetros esenciales y solo descansaba en su variación al azar cuando encontraba un comportamiento que mantenía los valores de esas variables críticas dentro de sus límites deseables. Las ideas de Ashby, desarrolladas en su famoso libro ``Design for a Brain'', dieron lugar al campo de estudio de los sistemas biológicos como sistemas homeostáticos y adaptativos en términos puramente de matemática de sistemas dinámicos.

Estas sugerentes ideas permiten operativizar nociones de automantenimiento biológico y fisiológico y llevarlas a un entorno de implementación de agentes artificiales. Como recordamos, nuestro objetivo era tener esquemas ultraestables en el ámbito de la conducta de un agente virtual. ¿Cuáles y cómo podrían ser los componentes de una organización neuronal-conductual para poder hacer una analogía entre la vida orgánica y la conducta de un agente programado?

Si, a partir de conceptos como la homeostasis, la supervivencia de la vida orgánica puede definirse como una red auto-mantenida de reacciones químicas alejadas del equilibrio, la conducta podría definirse análogamente como una red auto-mantenida de hábitos o estructuras neurodinámicas. El objetivo es extender las nociones fisiológicas a estructuras neuronales. Y, en este segundo caso, la conservación adaptativa de organizaciones neurodinámicas (en lugar de organizaciones fisiológicas) no vendría por mecanismos de tipo biológico sino mediante regulación sensomotora asociada a la conducta del agente (en lugar de basada en esquemas metabólicos).

\subsubsection{Controladores con plasticidad homeostática}
Esta analogía vida-conducta, que podríamos denominar interesada en nociones de ``homeostasis en sistemas nerviosos artificiales'', ha sido poco explorada\cite{DiPaolo_1, DiPaolo_2, DiPaolo_3, MathayomchanBeer,HoinvilleHenaff}. Tomaremos la misma aproximación que otros investigadores han hecho y que se basa en el uso de redes recurrentes continuas para modelar el sustrato de un controlador neuronal sobre el que se incorporan mecanismos basados en plasticidad hebbiana para garantizar que regímenes de activación de las redes en su conjunto, tienen características homeostáticas que les permiten lograr condiciones de ultraestabilidad.

Esta ``plasticidad homeostática'' no es solo un mecanismo formal sino que tiene evidencia empírica \cite{TurrigianoGG}, pues se ha observado que las neuronas buscan estabilidad en su frecuencia de disparo. El modo de conseguir estabilidad conductual, por tanto, parece exigir que los modos de estabilidad en la conducta aparezcan acoplados a los mecanismos de estabilidad de los parámetros sinápticos, de manera que cuando la estabilidad conductual se pierde induce inestabilidad sináptica hasta recuperar la estabilidad conductual.


El método para implementar plasticidad homeostática normalmente se apoya en los resultados de la teoría Hebbiana, que describe un mecanismo básico de plasticidad sináptica en el que el valor de una conexión sináptica se incrementa si las neuronas de ambos lados de dicha sinapsis se activan repetidas veces de forma simultánea. Introducida por Donald Hebb, en 1949, es también llamada ``regla de Hebb'' o ``Teoría de la Asamblea Celular'' y afirma que, en palabras del propio Hebb: <<Cuando el axón de una célula A está lo suficientemente cerca como para excitar a una célula B y repetidamente toma parte en la activación, ocurren procesos de crecimiento o cambios metabólicos en una o ambas células de manera que tanto la eficiencia de la célula A, como la capacidad de excitación de la célula B son aumentadas>>.

De manera menos rigurosa, la teoría se resume a menudo como que ``las células que se disparan juntas, permanecerán conectadas'', aunque esto es una simplificación y no debe tomarse literalmente, así como no representa con exactitud la declaración original de Hebb sobre cambios de la fuerza de conectividad en las células. Sin embargo, ese principio es que se suele utilizar en el ámbito del Aprendizaje automático y de los modelos en Neurocomputación. La teoría es comúnmente evocada para modelar algunos tipos de aprendizajes asociativos en redes neuronales artificiales en los que la activación simultánea de las células conduce a un pronunciado aumento de la fuerza sináptica (conocido como aprendizaje de Hebb).

\subsubsection{Controladores diseñados mediante computación evolutiva}
Hasta ahora hemos visto que sería posible implementar agentes cuya conducta esté basada en hábitos a partir de mecanismos neuronales con plasticidad hebbiana que garantizarían condiciones de homoestaticidad en la red como conjunto. Nos queda un último punto: seleccionar el tipo de red neuronal más adecuada para que puedan implementarse estos procesos y que suponga la base de un controlador de agentes virtuales. Para ello, nos fijamos en aspectos de la metodología denominada Computación Evolutiva, que se usa para el diseño de controladores neuronales evolutivos genéticamente determinados. Dichos controladores son aplicados a agentes (reales o virtuales) tipo Khepera para llevar a cabo tareas (por ejemplo, despliegue de trayectorias en navegación) dentro de un entorno no conocido previamente. Los controladores están basados en redes neuronales, que mediante evolución de parámetros con algoritmos genéticos, pueden ajustarse en forma on-line sin entrenamiento adicional a cambios del entorno, permitiendo un control apropiado de los este tipo de agentes robóticos. En los últimos años, en la mayoría de los experimentos de computación evolutiva son utilizados sistemas de control mediante redes neuronales con conexiones recurrentes, las cuales permiten que la red contemple aspectos de dinámica temporal\cite{JesperDario}.

La metodología empleada es la siguiente: una población de cromosomas artificiales es creada aleatoriamente y probada dentro del entorno del agente virtual. Concretamente, cada elemento de la población codifica el sistema de control de un robot. Cada robot es libre de actuar según al controlador utilizado, el cual fue genéticamente determinado, mientras es evaluado su desempeño (fitness) al realizar una tarea. Este proceso de generación de controladores robóticos y su evaluación en el entorno del robot, es llevado a cabo hasta satisfacer un criterio preestablecido (una fitness mínima que se supone aceptable por el programador) vinculado a la tarea a desarrollar.

En relación con la metodología mencionada, el desarrollo de tales controladores neuronales presupone la aceptación de algunos criterios, como los que a continuación se describen:

\begin{itemize}
  \item{Tipo de controlador neuronal: por lo general, el tipo de controlador utilizado es una red recurrente continua en el tiempo (Continuous Time Recurrent Neural Network. CTRNN, en su acrónimo en inglés). Estas son redes continuas cuyas neuronas están totalmente conectadas (se explotan los diversos bucles retroalimentados) con la peculiaridad matemática de que pueden aproximar cualquier sistema dinámico \cite{FunaYNaka}.}
  \item{Codificación genética de los controladores: se usan controladores genéticamente determinados (los pesos sinápticos son determinados genéticamente) y, en ocasiones, controladores con sinapsis adaptativas (los pesos sinápticos son siempre aleatorios y posteriormente modificados según reglas de adaptación de las sinapsis, las cuales son determinadas genéticamente).}
  \item{Modelo del robot: el tipo de robot a utilizar debe poseer un conjunto de sensores, los cuales permitan la navegación autónoma basada en ellos. En su mayoría, los trabajos presentados utilizan un agente virtual inspirado en el robot denominado Khepera.}
  \item{Tarea a realizar por el robot: diferentes clases de problemas pueden ser afrontados al momento de evaluar un controlador como el aquí expuesto. Principalmente, algunos de estos problemas se enmarcan en Aprendizaje por Refuerzo, análisis de las capacidades de adaptación a los cambios, estudio de la influencia de cambios en el entorno, navegación autónoma, etc. }
  \item{Hipótesis a probar: Existe un amplio espectro de áreas de investigación que pretenden ser estudiadas con el enfoque aquí presentado. Dichas áreas involucran desde aspectos ingenieriles hasta científicos\cite{Nolfi}, algunos de los cuales son: controladores bio-inspirados, modularidad y plasticidad neuronal, la perspectiva comportamental, aprendizaje, etc.}
\end{itemize}

\subsubsection{Capacidades sociales}
El impacto que tiene lo social en nuestras capacidades individuales ha sido objeto de disputa y análisis a lo largo de la historia en el campo de la Ciencia Cognitiva. Por un lado podemos distinguir entre aquellos autores que creen que las capacidades individuales se ven afectadas por un nivel social de normas, reglas y convenciones culturales sin que estas capacidades individuales varíen en su naturaleza. En este sentido, el nivel social vendría a complementar las capacidades cognitivas de los
individuos añadiendo un nivel cognitivo más que permitiría a los individuos organizarse, coordinarse y cooperar en el desarrollo de estrategias que satisfarían a todos a nivel individual. En este marco, las capacidades individuales tales como percepción, acción o memoria no se verían afectadas por los cambios que puedan existir en las dinámicas sociales que se generen. Por otro lado, otros autores defienden que las capacidades individuales sí se verían afectadas por las dinámicas sociales. En esta propuesta, el aspecto no sólo supondría un nivel nuevo que permite a los individuos organizarse y obtener una recompensa individual, sino que ese mismo nivel y esas mismas actividades afectarían a las capacidades de los individuos.

La primera aproximación, por la cual lo social simplemente añade un nivel de complejidad mayor a lo que ya viene dado tras años de evolución y que permanece impenetrable a lo social, se conocería como aproximación aditiva \cite{KernMoll}.
La segunda aproximación afirmaría que las capacidades sociales transforían a las capacidades individuales, de tal modo que la socialización generaría un efecto \textit{top-down} que cambia por completo la naturaleza de las capacidades individuales, lo que se conoce como una aproximación transformativa \cite{KernMoll}.

Existen muchas afirmaciones a favor y en contra de ambas aproximaciones pese a que los autores no utilicen explicitamente esta distinción. Por ejemplo, a favor del poder meramente aditivo de las capacidades sociales tenemos trabajos como los de Peters Carruthers \cite{Carruthers}. Entre los defensores del poder transformativo de lo social tenemos a autores como John McDowell\cite{McDowell}. Es posible ver este debate entre el carácter aditivo vs. el carácter transformativo como un reflejo actualizado del debate tradicional que se mantuvo sobre el origen de capacidades mentales asociadas a naturaleza o cultura.

\subsubsection{Conclusiones}
En conclusión, recopilando lo visto hasta ahora: (i) nos interesa el estudio de comportamiento de agentes basados en hábitos y exige incorporar en nuestros modelos nociones de plasticidad homeostática; (ii) los modos de implementación se basarán en modelos de aprendizaje hebbiano y en redes recurrentes continuas como controladores de los agentes virtuales programados; (iii) metodológicamente, se diseñará una arquitectura neuronal con parámetros libres, se codificará genéticamente y se evolucionará en relación con la tarea a realizar, lo que implica definir una función de evaluación o función de fitness, la cual determinará el grado de aceptación de cada controlador neuronal en el proceso evolutivo.
