\begin{center}
{\Large \bfseries Design and analysis of homeostatic adaptative networks}

\vspace{1cm}
{\Large \bfseries ABSTRACT}

\vspace{1.5cm}
\end{center}

Being able to give adaptability features present in living beings to artificial agents, would provide them with the
necessary skills to perform tasks for which they have not been trained for, or for which they have been trained
but in a changing environment or with uncertainty. In this project we are interested, from an engineering perspective, in
some of the self-regulating systems present in living organisms, known as homeostatic mechanisms, which are in the base of
certain adaptation features.

In particular, the starting point has been the design and implementation of an homeostatic agent based on continuous time recurent
neuronal networks with fototaxis features (ability to search and get closer to a lightsource). This agent will be the
best candidate within a certain population, selected by a genetic algorithm. Which is interesting from the agent is that he
develops homeostatic behaviour by plasticity mechanisms that let him modify his internal variables in order to reach a
stable state and to keep himself in it even if affected by external disturbances.

Once the agent with fototaxis behaviour and plasticity properties is obtained, we will give him certain features to let social behaviours appear,
with the goal of analizing how does he interact in situations where more than one agent is involved. This social behaviour will be added following two
different approaches. The first one assumes that agent features are additive. This means that the agent, once being
able to orientate and move closer to a lightsource (or food), would need a new neuronal structure to encode a new feature (social
coordination in this case). The system will be modeled with two different mechanisms in the controller, one for navigation
and one for the social part. The sencond one defends that social behaviour is a structural feature of the system. Therefore,
a new feature breaks throught previous features, restructuring the neuronal controller of the agent as a whole. The objective
is to compare this two approaches and get conclusions about it by executing experiments in which various agents have to interact in
a social way to archieve an objective.
