\chapter{Introducción}
\pagenumbering{arabic}
Incorporar comportamiento adaptativo en agentes artificales es uno de los medios esenciales para que estos puedan desenvolverse en entornos con cambios o incertidumbre de manera satisfactoria.

En ingeniería, uno de los pioneros trabajos en este ámbito se debe a William Ashby ('A Design for a Brain'), en el que se interesó por investigar como funcionan los mecanismos de
adaptación en sistemas vivos. La tesis de Ashby fue que todos los organismos tienen ciertas variables esenciales, que deben mantenerse dentro de unos límites, y que a menudo están
vinculadas entre sí, provocando que cambios en algunas de ellas puedan afectar a las demás. El estado de supervivencia del organismo ocurre cuando un comportamiento no mueve ninguna variable esencial
fuera de sus límites. Al conjunto de mecanismos que permiten esta estabilidad se lo conoce como regulación homeostática. En lugar de atribuir a los organismos conductas intencionales de búsqueda de objetivos,
Ashby exploró estos mecanismos que permitían generar un comportamiento adaptativo autoinducido mediante el mantenimiento de la estabilidad interna de sus variables esenciales.

Trabajos recientes en neurociencia han mostrado que existen mecanismos reguladores homeostáticos de eficacia sináptica en el cerebro. Estos mecanismos entre neuronas están impulsados por la
necesidad de adaptarse a los cambios que dependen de una actividad, al tiempo que mantienen la estabilidad interna.

Dotar a un agente artificial de propiedades de adaptación homeostática le permitiría adaptarse y resolver problemas para los que inicialmente no fue entrenado, permitiéndole alterar sus variables
internas hasta obtener una estabilidad completa de su sistema al mismo tiempo que ejecuta las tareas encomendadas.

\section{Contexto y motivación}
Me planteo estudiar el comportamiento..... para ...... TODO TODO TODO

\section{Objetivos}
En este trabajo de fin de grado se pretenden aplicar y analizar las ideas de
ultrastabilidad de William Ashby en agentes artificiales. Dichos agentes serán diseñados mediante controladores
basados en redes neuronales recurrentes a las que se les impondrá una regulación
homeostática de su actividad sináptica (las neuronas tendrán funciones de activación
adecuadas para generar estas estructuras de estabilidad). Los agentes serán evolucionados
para solucionar un problema de fototaxis (búsqueda y acercamiento a una fuente de luz).
Una vez programados agentes artificiales homeostáticos se buscará analizar:

\begin{itemize}
\item Si los agentes pueden adaptarse a cambios en las reglas del entorno, aunque no hayan
evolucionado específicamente para resolver nuevas tareas. Se buscará probar que, tal
como ocurre en organismos vivos, estos mecanismos de adaptación facilitan
implícitamente el mantenimiento de estabilidad y el desarrollo de nuevos
comportamientos.
\item  Si en entornos sociales de agentes artificiales, donde los comportamientos para resolver
problemas se basen en estrategias colectivas, los patrones de estabilidad son más
robustos. Para ello se someterán las configuraciones obtenidas a estímulos ruidosos o
perturbaciones, y mediremos el grado de estabilidad de los patrones homeostáticos.
\end{itemize}

\section{Estructura}
La memoria cuenta con un primer apartado de introducción, seguido por otro del estado del arte, en los que se explica el contexto del problema a resolver, los principales objetivos,
motivaciones y situación actual del ámbito en el que se desarrolla el trabajo. A continuación se encuentra el apartado de diseño e implementación, en el cual se habla sobre las distintas
herramientas y técnicas utilizadas para la realización del trabajo. Por último, en la sección de experimentación se enuncian y se analizan los dos procesos que se han ejecutado para la
realización de los dos objetivos anteriormente descritos.

En los anexos se puede observar el tiempo invertido en este proyecto junto con las tareas las que se dedicó. Además,
se incluye el código correspondiente en el lenguaje de programación Python desarollado para la obtención de los resultados necesarios.
